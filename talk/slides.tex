\documentclass[xetex,mathserif,serif]{beamer}
\usetheme{metropolis}

\title{On the Edge-length Ratio of Outerplanar Graphs}
\author{Aleksandr Lisianoi}
\institute{Technische Universität Wien}
\date{\today}


\begin{document}
\maketitle

\begin{frame}
  \frametitle{Main result}
  \begin{theorem}
    The planar edge-length ratio of an outerplanar graph is strictly less than 2. Also, for any given real positive number \(\varepsilon\), there exists an outerplanar graph whose planar edge-length ratio is greater than \(2 - \varepsilon\).
    \end{theorem}
  \end{frame}

\begin{frame}[standout]
  Thank you!
  \end{frame}

\begin{frame}
    \frametitle{Backup slide}

    \begin{definition}[k-vertex-connected graph]
      A connected graph \(G\) is a k-vertex-connected (k-connected) if it has more than k vertices and remains connected if fewer than k vertices are removed.
    \end{definition}
  \end{frame}

  \begin{frame}
    \frametitle{Backup slide: planar and outerplanar graphs}

    \begin{definition}[planar graph]
      A graph \(G\) is a planar graph if it can embedded in the plane, i.e. it can be drawn on the plane in such a way that its edges intersect only at their endpoints.
    \end{definition}

    \begin{definition}[planar graph: forbidden minors]
      A finite graph \(G\) is planar if and only if its minors include neither \(K_{5,5}\) nor \(K_{3, 3}\).
    \end{definition}

    \begin{definition}[outerplanar graph]
      A graph \(G\) is an outerplanar graph if it has a planar drawing for which all vertices belong to the outer edge.
    \end{definition}

    \begin{definition}[outerplanar graph: forbidden minors]
      A finite graph \(G\) is outerplanar if and only if its minors include neither \(K_{5, 5}\), nor \(K_{3, 3}\), nor \(K_4\) nor \(K_{2, 3}\).
    \end{definition}
  \end{frame}
% etc
\end{document}
