\documentclass[xetex,mathserif,serif]{beamer}
\usetheme{metropolis}

\usepackage{svg}
\usepackage[
  backend=biber
  , style=numeric
  , sortlocale=en_US
  , url=false
  , doi=true
  , eprint=false
  , giveninits=false
  , maxbibnames=20
  , maxnames=20
  , maxcitenames=4
]{biblatex}
\addbibresource{references/references.bib}

\graphicspath{./figures/}

\title{On the Edge-length Ratio of Outerplanar Graphs}
\author{Aleksandr Lisianoi}
\institute{Technische Universität Wien}
\date{\today}


\begin{document}

\maketitle

\section{Introduction}

\begin{frame}
  \frametitle{Acknowledgement}

  This presentation is based on the following work:

  \fullcite{2017-lazard-ratio}
\end{frame}

\begin{frame}
  \frametitle{Motivation}

  TODO
  \end{frame}

\section{Preliminary Definitions}

\begin{frame}
  \frametitle{Outerplanar graphs}

    \begin{definition}[outerplanar graph]
    A graph \(G\) is an outerplanar graph if it has a planar drawing for which all vertices belong to the outer edge.
  \end{definition}

  \begin{definition}[outerplanar graph: forbidden minors]
    A finite graph \(G\) is outerplanar if and only if its minors include neither \(K_4\) nor \(K_{2, 3}\).
  \end{definition}
\end{frame}

\section{Main Result}

\begin{frame}
  \frametitle{Main result}
  \begin{theorem}
    The planar edge-length ratio of an outerplanar graph is strictly less than 2. Also, for any given real positive number \(\varepsilon\), there exists an outerplanar graph whose planar edge-length ratio is greater than \(2 - \varepsilon\).
    \end{theorem}
\end{frame}

\begin{frame}
  \frametitle{Chain definition}
  A sequence \(T_s, T_{s+1}, \dots, T_t\) of triangles from \(G\), s.t.
  \begin{enumerate}
  \item \(s \leq 0 \leq t\)
  \item \(T_0\) has an outer edge from \(G\) (its vertices labeled 0) and a third vertex (labeled 1).
  \item \(\forall i\in [1, t]\): the vertices of \(T_i\) are labeled \(\left\{i - 1, i, i + 1\right\}\). Triangles \(T_i\) and \(T_{i - 1}\) share an edge \((i, i - 1)\).
  \item \(\forall i\in [s, -1]\): the vertices of \(T_i\) are labeled \(\left\{i, i + 1, i + 2\right\}\). Triangles \(T_i\) and \(T_{i + 1}\) share an edge \((i + 1, i + 2)\).
    \end{enumerate}
  \end{frame}

\begin{frame}
  \frametitle{Chain drawing lemma}
  \begin{lemma}
    Given a chain C with n vertices, an external edge e of C, a segment s of length 1 in the plane and a direction d such that the (smaller) angle between s and d is \(\theta < \theta_0 = \arccos(1/4)\approx 75,5^{\circ}\), there exists a planar straight-line drawing of C such that:
    \begin{enumerate}
    \item The drawing is completely contained within the strip \(S(s, d)\).
    \item For each external edge \(e'\) of C:
      \begin{enumerate}
      \item It has length 1.
      \item It is not parallel to d and forms an angle less than \(\theta_0\) with it.
      \item Strip \(S(e', d)\) is empty.
      \end{enumerate}
    \item Each non-external edge of C has length greater than 1/2.
    \end{enumerate}
  \end{lemma}
\end{frame}

\begin{frame}[standout]
  Thank you!
\end{frame}

\begin{frame}[fragile]{References}
\printbibliography
\end{frame}

\appendix

\section{Backup Slides}

\begin{frame}
  \frametitle{Backup slide: connected graphs}

  \begin{definition}[k-vertex-connected graph]
    A connected graph \(G\) is a k-vertex-connected (k-connected) if it has more than k vertices and remains connected if fewer than k vertices are removed.
  \end{definition}
\end{frame}

\begin{frame}
  \frametitle{Backup slide: planar and outerplanar graphs}

  \begin{definition}[planar graph]
    A graph \(G\) is a planar graph if it can embedded in the plane, i.e. it can be drawn on the plane in such a way that its edges intersect only at their endpoints.
  \end{definition}

  \begin{definition}[planar graph: forbidden minors]
    A finite graph \(G\) is planar if and only if its minors include neither \(K_{5,5}\) nor \(K_{3, 3}\).
  \end{definition}

  \begin{definition}[outerplanar graph]
    A graph \(G\) is an outerplanar graph if it has a planar drawing for which all vertices belong to the outer edge.
  \end{definition}

  \begin{definition}[outerplanar graph: forbidden minors]
    A finite graph \(G\) is outerplanar if and only if its minors include neither \(K_4\) nor \(K_{2, 3}\).
  \end{definition}
\end{frame}

\begin{frame}
  \frametitle{Backup slide: outerplanar graphs}

  \begin{definition}[maximal outerplanar graph]
    An outerplanar graph \(G\) is a maximal outerplanar graph if no more edges can be added while preserving outerplanarity.
  \end{definition}
\end{frame}

\begin{frame}
  \frametitle{Backup slide: dual graphs}

  \begin{definition}[dual graph]
    Given a plane graph \(G\), its dual graph \(G^*\) is a plane graph that has a vertex for each face of \(G\). The dual graph \(G^*\) has an edge whenever two faces of \(G\) are separated from each other by an edge, and a self-loop when the same face appears on both sides of an edge.
  \end{definition}
\end{frame}

\begin{frame}
  \frametitle{Backup slide: dual graphs (example)}

  \begin{figure}
    \begin{center}
      \includesvg[width=0.6\textwidth]{figures/nonisomorphic-dual-graphs}
    \end{center}
    \caption{Two non-isomorphic dual graphs, \href{https://en.wikipedia.org/wiki/Dual_graph\#/media/File:Noniso_dual_graphs.svg}{image source}.}
  \end{figure}
  \end{frame}

\begin{frame}
  \frametitle{Backup slide: series-parallel graphs}
  See lecture 3
\end{frame}


% etc
\end{document}
