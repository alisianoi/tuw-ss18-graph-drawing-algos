\documentclass[xetex,mathserif,serif]{beamer}
\usetheme{metropolis}

\usepackage{svg}
\usepackage[
  backend=biber
  , style=numeric
  , sortlocale=en_US
  , url=false
  , doi=true
  , eprint=false
  , giveninits=false
  , maxbibnames=20
  , maxnames=20
  , maxcitenames=4
]{biblatex}
\addbibresource{references/references.bib}

\graphicspath{./figures/}

\title{On the Edge-length Ratio of Outerplanar Graphs}
\author{Aleksandr Lisianoi}
\institute{Technische Universität Wien}
\date{\today}


\begin{document}

\tikzstyle{vertex} = [
  circle
  , fill=black
  , minimum size=5pt
  , inner sep=0pt
]
\tikzstyle{fvertex} = [
  circle
  , fill=white
  , minimum size=5pt
  , inner sep=0pt
]
\tikzstyle{edge} = [
  draw
  , thick
  , -
]
\tikzstyle{dedge} = [
  draw
  , very thick
  , ->
]


\maketitle

\section{Introduction}

\begin{frame}
  \frametitle{Acknowledgement}

  This presentation is based on the following work:

  \fullcite{2017-lazard-ratio}
\end{frame}

\begin{frame}
  \frametitle{Motivation}

  \begin{itemize}
  \item Does a graph with prescribed edge lengths admit a planar straight-line drawing?
    \pause
  \item Does a two- or three-connected planar graph admit a unit-length planar straight-line drawing?
    \pause
  \item Does a degree four tree have a planar straight-line drawing with vertices at grid positions and edges of the same length?
  \end{itemize}
  \pause
  \begin{center}
    All these problems are NP-hard.
  \end{center}

\end{frame}

\begin{frame}
  \frametitle{Relax problem requirements}
  \begin{itemize}
  \item Choose a specific graph family.
  \item Accept ``close enough'' quality criteria.
    \end{itemize}
\end{frame}

\section{Preliminary Definitions}

\begin{frame}
  \frametitle{Outerplanar graphs}

    \begin{definition}[outerplanar graph]
    A graph \(G\) is an outerplanar graph if it has a planar drawing for which all vertices belong to the outer edge.
  \end{definition}

  \begin{definition}[outerplanar graph: forbidden minors]
    A finite graph \(G\) is outerplanar if and only if its minors include neither \(K_4\) nor \(K_{2, 3}\).
  \end{definition}
\end{frame}

\begin{frame}
  \frametitle{Outerplanar graphs: negative examples}

    \begin{figure}
      \begin{tikzpicture}[scale=1,auto,swap]
        \onslide<4->
        \node[vertex] (5) at (5, 1)  {};
        \node[vertex] (6) at (6, 1)  {};
        \node[vertex] (7) at (7, 1)  {};
        \node[vertex] (8) at (5.5, 3)  {};
        \node[vertex] (9) at (6.5, 3)  {};

        \path[edge] (5) -- (8);
        \path[edge] (5) -- (9);
        \path[edge] (6) -- (8);
        \path[edge] (6) -- (9);
        \path[edge] (7) -- (8);
        \path[edge] (7) -- (9);

        \onslide<3->
        \node[vertex] (0) at (3, 1)  {};
        \node[vertex] (1) at (2, 2)  {};
        \node[vertex] (2) at (3, 2)  {};
        \node[vertex] (3) at (4, 2)  {};
        \node[vertex] (4) at (3, 3)  {};

        \path[edge] (0) -- (1);
        \path[edge] (0) -- (2);
        \path[edge] (0) -- (3);
        \path[edge] (1) -- (4);
        \path[edge] (2) -- (4);
        \path[edge] (3) -- (4);

        \onslide<2->
        \node[vertex] (14) at (5, 4)   {};
        \node[vertex] (15) at (7, 4)   {};
        \node[vertex] (16) at (6.5, 5) {};
        \node[vertex] (17) at (5.5, 5) {};
        \node[vertex] (18) at (6, 6)   {};

        \path[edge] (14) -- (16);
        \path[edge] (14) -- (17);
        \path[edge] (15) -- (16);
        \path[edge] (15) -- (17);
        \path[edge] (16) -- (18);
        \path[edge] (17) -- (18);

        \onslide<1->
        \node[vertex] (10) at (2, 4)  {};
        \node[vertex] (11) at (4, 4)  {};
        \node[vertex] (12) at (3, 4.8)  {};
        \node[vertex] (13) at (3, 6)  {};

        \path[edge] (10) -- (11);
        \path[edge] (10) -- (12);
        \path[edge] (10) -- (13);
        \path[edge] (11) -- (12);
        \path[edge] (11) -- (13);
        \path[edge] (12) -- (13);


    \end{tikzpicture}
    \caption{\(K_4\) (top left) and embeddings of \(K_{2, 3}\) (all others)}
  \end{figure}

  \end{frame}

\begin{frame}
  \frametitle{Outerplanar graphs: positive example}

  \begin{figure}
    \begin{tikzpicture}[scale=1,auto,swap]
      \node[vertex] (0) at ( 0, 1.73)  {};
      \node[vertex] (1) at ( 2, 1.73)  {};
      \node[vertex] (2) at (-3, 0)     {};
      \node[vertex] (3) at (-1, 0)     {};
      \node[vertex] (4) at ( 1, 0)     {};
      \node[vertex] (5) at ( 3, 0)     {};
      \node[vertex] (6) at (-2, -1.73) {};
      \node[vertex] (7) at ( 0, -1.73) {};
      \node[vertex] (8) at ( 2, -1.73) {};

      \path[edge] (0) -- (1);
      \path[edge] (0) -- (3);
      % \path[edge] (0) -- (4);
      % \path[edge] (1) -- (4);
      \path[edge] (1) -- (5);
      \path[edge] (2) -- (3);
      \path[edge] (2) -- (6);
      \path[edge] (3) -- (4);
      % \path[edge] (3) -- (6);
      \path[edge] (3) -- (7);
      \path[edge] (4) -- (5);
      \path[edge] (4) -- (7);
      \path[edge] (4) -- (8);
      \path[edge] (5) -- (8);
      \path[edge] (6) -- (7);
    \end{tikzpicture}
    \caption{A non-trivial outerplanar graph}
  \end{figure}
\end{frame}

\begin{frame}
  \frametitle{Outerplanar graphs: maximal example}

  \begin{figure}
    \begin{tikzpicture}[scale=1,auto,swap]
      \node[vertex] (0) at ( 0, 1.73)  {};
      \node[vertex] (1) at ( 2, 1.73)  {};
      \node[vertex] (2) at (-3, 0)     {};
      \node[vertex] (3) at (-1, 0)     {};
      \node[vertex] (4) at ( 1, 0)     {};
      \node[vertex] (5) at ( 3, 0)     {};
      \node[vertex] (6) at (-2, -1.73) {};
      \node[vertex] (7) at ( 0, -1.73) {};
      \node[vertex] (8) at ( 2, -1.73) {};

      \path[edge] (0) -- (1);
      \path[edge] (0) -- (3);
      \path[edge] (0) -- (4);
      \path[edge] (1) -- (4);
      \path[edge] (1) -- (5);
      \path[edge] (2) -- (3);
      \path[edge] (2) -- (6);
      \path[edge] (3) -- (4);
      \path[edge] (3) -- (6);
      \path[edge] (3) -- (7);
      \path[edge] (4) -- (5);
      \path[edge] (4) -- (7);
      \path[edge] (4) -- (8);
      \path[edge] (5) -- (8);
      \path[edge] (6) -- (7);
    \end{tikzpicture}
    \caption{Maximal outerplanar graph}
  \end{figure}
\end{frame}

\section{Main Result}

\begin{frame}
  \frametitle{Main result}
  \begin{theorem}
    The planar edge-length ratio of an outerplanar graph is strictly less than 2. Also, for any given real positive number \(\varepsilon\), there exists an outerplanar graph whose planar edge-length ratio is greater than \(2 - \varepsilon\).
    \end{theorem}
\end{frame}

\begin{frame}
  \frametitle{Chain definition}
  A sequence \(T_s, T_{s+1}, \dots, T_t\) of triangles from \(G\), s.t.
  \begin{enumerate}
  \item \(s \leq 0 \leq t\)
  \item \(T_0\) has an outer edge from \(G\) (its vertices labeled 0) and a third vertex (labeled 1).
  \item \(\forall i\in [1, t]\): the vertices of \(T_i\) are labeled \(\left\{i - 1, i, i + 1\right\}\). Triangles \(T_i\) and \(T_{i - 1}\) share an edge \((i, i - 1)\).
  \item \(\forall i\in [s, -1]\): the vertices of \(T_i\) are labeled \(\left\{i, i + 1, i + 2\right\}\). Triangles \(T_i\) and \(T_{i + 1}\) share an edge \((i + 1, i + 2)\).
  \end{enumerate}

  \pause
  \emph{Property:} this prohibits fans with more than 3 triangles for all vertices, except \(v_1\).
\end{frame}

\begin{frame}
  \frametitle{Chain building example}

  \begin{figure}
    \begin{tikzpicture}[scale=1,auto,swap]

      \onslide<1->
      \node[fvertex] (50) at ( 0,     2) {\(v_0\)};
      \node[fvertex] (51) at (-1.1, 0.3) {\(v_0\)};

      \onslide<2->
      \fill[green!30] (0,1.73) -- (-1,0) -- (1,0) -- cycle;
      \node[fvertex,fill=green!30] (20) at (0, 0.6) {\(T_0\)};

      \node[fvertex] (52) at (1, -0.4) {\(v_1\)};

      \onslide<3->
      \fill[gray!30] (0,1.73) -- (-1,0) -- (1,0) -- cycle;
      \node[fvertex,fill=gray!30] (20) at (0, 0.6) {\(T_0\)};

      \fill[green!30] (0,1.73) -- (2,1.73) -- (1,0) -- cycle;
      \node[fvertex,fill=green!30] (21) at (1, 1) {\(T_1\)};

      \node[fvertex] (53) at (2, 2) {\(v_2\)};

      \onslide<4->
      \fill[gray!30] (0,1.73) -- (-1,0) -- (1,0) -- cycle;
      \node[fvertex,fill=gray!30] (20) at (0, 0.6) {\(T_0\)};

      \fill[gray!30] (0,1.73) -- (2,1.73) -- (1,0) -- cycle;
      \node[fvertex,fill=gray!30] (21) at (1, 1) {\(T_1\)};

      \fill[green!30] (1,0) -- (2,1.73) -- (3,0) -- cycle;
      \node[fvertex,fill=green!30] (22) at (2, 0.6) {\(T_2\)};

      \node[fvertex] (54) at (3.3, 0) {\(v_3\)};

      \onslide<5->
      \fill[gray!30] (0,1.73) -- (-1,0) -- (1,0) -- cycle;
      \node[fvertex,fill=gray!30] (20) at (0, 0.6) {\(T_0\)};

      \fill[gray!30] (0,1.73) -- (2,1.73) -- (1,0) -- cycle;
      \node[fvertex,fill=gray!30] (21) at (1, 1) {\(T_1\)};

      \fill[gray!30] (1,0) -- (2,1.73) -- (3,0) -- cycle;
      \node[fvertex,fill=gray!30] (22) at (2, 0.6) {\(T_2\)};

      \fill[green!30] (-1,0) -- (1,0) -- (0,-1.73) -- cycle;
      \node[fvertex,fill=green!30] (23) at (0, -0.6) {\(T_{-1}\)};

      \node[fvertex] (55) at (0, -2) {\(v_{-1}\)};

      \onslide<6->
      \fill[gray!30] (0,1.73) -- (-1,0) -- (1,0) -- cycle;
      \node[fvertex,fill=gray!30] (20) at (0, 0.6) {\(T_0\)};

      \fill[gray!30] (0,1.73) -- (2,1.73) -- (1,0) -- cycle;
      \node[fvertex,fill=gray!30] (21) at (1, 1) {\(T_1\)};

      \fill[gray!30] (1,0) -- (2,1.73) -- (3,0) -- cycle;
      \node[fvertex,fill=gray!30] (22) at (2, 0.6) {\(T_2\)};

      \fill[gray!30] (-1,0) -- (1,0) -- (0,-1.73) -- cycle;
      \node[fvertex,fill=gray!30] (23) at (0, -0.6) {\(T_{-1}\)};

      \fill[green!30] (-1,0) -- (0,-1.73) -- (-2,-1.73) -- cycle;
      \node[fvertex,fill=green!30] (24) at (-1, -1) {\(T_{-2}\)};

      \node[fvertex] (56) at (-2, -2) {\(v_{-2}\)};

      \onslide<7->
      \fill[gray!30] (0,1.73) -- (-1,0) -- (1,0) -- cycle;
      \node[fvertex,fill=gray!30] (20) at (0, 0.6) {\(T_0\)};

      \fill[gray!30] (0,1.73) -- (2,1.73) -- (1,0) -- cycle;
      \node[fvertex,fill=gray!30] (21) at (1, 1) {\(T_1\)};

      \fill[gray!30] (1,0) -- (2,1.73) -- (3,0) -- cycle;
      \node[fvertex,fill=gray!30] (22) at (2, 0.6) {\(T_2\)};

      \fill[gray!30] (-1,0) -- (1,0) -- (0,-1.73) -- cycle;
      \node[fvertex,fill=gray!30] (23) at (0, -0.6) {\(T_{-1}\)};

      \fill[gray!30] (-1,0) -- (0,-1.73) -- (-2,-1.73) -- cycle;
      \node[fvertex,fill=gray!30] (24) at (-1, -1) {\(T_{-2}\)};

      \onslide<1->
      \node[vertex] (0) at ( 0, 1.73)  {};
      \node[vertex] (1) at ( 2, 1.73)  {};
      \node[vertex] (2) at (-3, 0)     {};
      \node[vertex] (3) at (-1, 0)     {};
      \node[vertex] (4) at ( 1, 0)     {};
      \node[vertex] (5) at ( 3, 0)     {};
      \node[vertex] (6) at (-2, -1.73) {};
      \node[vertex] (7) at ( 0, -1.73) {};
      \node[vertex] (8) at ( 2, -1.73) {};

      \path[edge] (0) -- (1);
      \path[edge] (0) -- (3);
      \path[edge] (0) -- (4);
      \path[edge] (1) -- (4);
      \path[edge] (1) -- (5);
      \path[edge] (2) -- (3);
      \path[edge] (2) -- (6);
      \path[edge] (3) -- (4);
      \path[edge] (3) -- (6);
      \path[edge] (3) -- (7);
      \path[edge] (4) -- (5);
      \path[edge] (4) -- (7);
      \path[edge] (4) -- (8);
      \path[edge] (5) -- (8);
      \path[edge] (6) -- (7);

    \end{tikzpicture}
  \end{figure}
\end{frame}

\begin{frame}
  \frametitle{Half-inifinite strip}

  \begin{figure}
    \begin{tikzpicture}[scale=1,auto,swap]

      \fill[gray!30] (0, 3) -- (4, 5) -- (4, 2) -- (0, 0) -- cycle;

      \node[vertex] (0) at (0, 3) {};
      \node[vertex] (1) at (0, 0) {};

      \node[vertex] (2) at (-3, -1) {};
      \node[vertex] (3) at (-1,  0) {};

      \path[edge] (0) -- node{\(s\)} (1);

      \draw[dedge] (2) -- node{\(\vec{\bf{d}}\)} (3);

      \draw[dedge] (0) -- (2, 4);
      \draw[dedge] (1) -- (2, 1);

      \draw[very thick, dashed] (0) -- (4, 5);
      \draw[very thick, dashed] (1) -- (4, 2);

      \node[fvertex,fill=gray!30] (10) at (2, 2.5) {\(S(s, \vec{\bf{d}})\)};
    \end{tikzpicture}
  \end{figure}
\end{frame}

\begin{frame}
  \frametitle{Chain drawing lemma}
  \begin{lemma}
    Given a chain C with n vertices, an external edge e of C, a segment s of length 1 in the plane and a direction d s.t. the (smaller) angle between s and d is \(\theta < \theta_0 = \arccos(1/4)\approx 75.5^{\circ}\), there exists a planar straight-line drawing of C such that:
    \begin{enumerate}
    \item The drawing is completely contained within the strip \(S(s, d)\).
    \item For each external edge \(e'\) of C:
      \begin{enumerate}
      \item It has length 1.
      \item It is not parallel to d and forms an angle less than \(\theta_0\) with it.
      \item Strip \(S(e', d)\) is empty.
      \end{enumerate}
    \item Each non-external edge of C has length greater than 1/2.
    \end{enumerate}
  \end{lemma}
\end{frame}

\begin{frame}
  \begin{figure}
    \begin{tikzpicture}[scale=0.5,auto,swap]
      \node[fvertex] (0) at (0, -0.7)  {\(v_{0}^{-}\)};
      \node[fvertex] (1) at (-3, 4.7) {\(v_{0}^{+}\)};
      \node[fvertex] (2) at ( 2, 4.5) {\(v_{2}\)};
      \draw[dashed] ( 0, 0) circle (2.5);
      \draw[dashed] (-3, 4) circle (2.5);
      \draw[dashed] ( 2, 4) circle (2.5);

      \path[edge] (0, 0)  -- (-3, 4);
      \path[edge] (0, 0)  -- (10, 0);
      \path[edge] (-3, 4) -- (10, 4);
    \end{tikzpicture}
  \end{figure}
\end{frame}

\begin{frame}[standout]
  Thank you!
\end{frame}

\begin{frame}[fragile]{References}
\printbibliography
\end{frame}

\appendix

\section{Backup Slides}

\begin{frame}
  \frametitle{Backup slide: connected graphs}

  \begin{definition}[k-vertex-connected graph]
    A connected graph \(G\) is a k-vertex-connected (k-connected) if it has more than k vertices and remains connected if fewer than k vertices are removed.
  \end{definition}
\end{frame}

\begin{frame}
  \frametitle{Backup slide: planar and outerplanar graphs}

  \begin{definition}[planar graph]
    A graph \(G\) is a planar graph if it can embedded in the plane, i.e. it can be drawn on the plane in such a way that its edges intersect only at their endpoints.
  \end{definition}

  \begin{definition}[planar graph: forbidden minors]
    A finite graph \(G\) is planar if and only if its minors include neither \(K_{5,5}\) nor \(K_{3, 3}\).
  \end{definition}

  \begin{definition}[outerplanar graph]
    A graph \(G\) is an outerplanar graph if it has a planar drawing for which all vertices belong to the outer edge.
  \end{definition}

  \begin{definition}[outerplanar graph: forbidden minors]
    A finite graph \(G\) is outerplanar if and only if its minors include neither \(K_4\) nor \(K_{2, 3}\).
  \end{definition}
\end{frame}

\begin{frame}
  \frametitle{Backup slide: outerplanar graphs}

  \begin{definition}[maximal outerplanar graph]
    An outerplanar graph \(G\) is a maximal outerplanar graph if no more edges can be added while preserving outerplanarity.
  \end{definition}
\end{frame}

\begin{frame}
  \frametitle{Backup slide: dual graphs}

  \begin{definition}[dual graph]
    Given a plane graph \(G\), its dual graph \(G^*\) is a plane graph that has a vertex for each face of \(G\). The dual graph \(G^*\) has an edge whenever two faces of \(G\) are separated from each other by an edge, and a self-loop when the same face appears on both sides of an edge.
  \end{definition}
\end{frame}

\begin{frame}
  \frametitle{Backup slide: dual graphs (example)}

  \begin{figure}
    \begin{center}
      \includesvg[width=0.6\textwidth]{figures/nonisomorphic-dual-graphs}
    \end{center}
    \caption{Two non-isomorphic dual graphs, \href{https://en.wikipedia.org/wiki/Dual_graph\#/media/File:Noniso_dual_graphs.svg}{image source}.}
  \end{figure}
  \end{frame}

\begin{frame}
  \frametitle{Backup slide: series-parallel graphs}
  See lecture 3
\end{frame}


% etc
\end{document}
